\section{Conclusion}\label{sec:conclusion}
In this paper we presented several statistics to help tune and validate the design of a Kalman filter for GPS-based localization.  We also proposed an extended model that considered per-satellite biases as part of the estimated state.

Several statistical hypothesis tests were very useful in the design of the filter.  For example, the normalized innovation squared normality test was extremely useful as it allowed us to rapidly scale covariance matrices on a per dataset basis.  The primary drawback is that since we had a model mismatch, it was very difficult to tune the system to achieve success on the test.  Figure~\ref{fig:nis} shows that the acceptable bad is extremely small for a 99\% confidence interval given that the normalized innovation squared values have such high variance and that the normalized innovation squared values undergo a change throughout the experiment.  However, Figure~\ref{fig:nis_stationary} shows that for a stationary receiver, the assumptions match quite well.  The smooth trends in these normalized innovation squared plots indicate that it might be beneficial to explore more dynamic forms of the process and measurement noise covariance matrices.  An attempt was made to dynamically scale $\alpha_k$ for the process covariance, but it was difficult to prevent the system from going unstable.  Since scaling the measurement covariance matrix offered the ability to pass the global consistency tests, then perhaps more focus should be placed on automatically scaling this matrix under the assumption that $\sigma_{PR}$ and $\sigma_D$ and poorly estimated as-is.

The Ljung-Box test for multivariate independence was not very useful.  All of the data used in the experiments intuitively had correlations and the statistical hypothesis test reflected that.  As explored in lab, it's possible to find an appropriate lag at which measurements are no longer correlated, but for a mobile receiver, it's probably the case that it's safer to break the independence assumption rather than only take one sample every 50 seconds.  Despite violating the independence assumptions, the filter qualitatively performed quite well localizing the vehicle in the driving corridor despite largely erroneous navigation solutions.

It's unclear whether the bias-adjusted time-varying Kalman filter formulation was a step in the right direction.  While its impact was validated in terms of reduced position error on the stationary dataset, it also caused the jerk normality test to fail suggesting that an additional systematic bias was introduced into the filter.  For mobile datasets, it offered little difference which is why results were omitted for those datasets in this paper.  A more principled approach to addressing the slowly varying errors in GPS measurements might be to instead use the Extended Kalman filter to model this as a non-linear system, extend the measurements to include raw GPS observables, error correction factors, etc. and track per satellite states in the filter's state vector.  In some sense, this current implementation is nearly using an EKF except all of the non-linear parts have been factored away into \texttt{solveposvelod}.  All of the work done by \texttt{solveposvelod} could instead be hoisted into the Jacobian of the EKF. 
